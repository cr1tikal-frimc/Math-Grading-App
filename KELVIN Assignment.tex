\documentclass[12pt,a4paper]{article}
\usepackage[utf8]{inputenc}
\usepackage{amsmath}
\usepackage{amsfonts}
\usepackage{amssymb}
\usepackage{natbib}
\usepackage{makeidx}
\usepackage{graphicx}
\usepackage{xcolor}
\usepackage{setspace}
\begin{document}

\begin{center}
	Dongzoola Dongmolar Kelvin\\
\end{center}
\begin{center}
\underline{\textbf{ASSIGNMENT SOLUTION}}
\end{center}\vspace{5mm}
\textbf{Question (1)}\\
\begin{enumerate}
\item Given $A = (x,y)i$ and $B = (x,y) 
j$ such that $A , B \in E$

	\item $\nabla \cdot  E =$
	\begin{align*}
	&\langle \dfrac{\partial}{\partial x} , \dfrac{\partial}{\partial y} \rangle \cdot  \langle
	Ai, Bj \rangle
	& = \dfrac{\partial A}{\partial x} + \dfrac{\partial B}{\partial y} =
	\dfrac{\partial (x,y)}{\partial x} + \dfrac{\partial (x,y)}{\partial y} \\
	&= 1 + 1 \\
	&= 2
	\end{align*}
	$\nabla \cdot ~ E $ is positive
	\item $\nabla \times ~ E =$
	\begin{align*}
	&= \langle \dfrac{\partial}{\partial x} , \dfrac{\partial}{\partial y} \rangle \times ~ \langle
	A\hat{i}, B\hat{j} \rangle \\
	&=\begin{vmatrix}
	\dfrac{\partial}{\partial x} & \dfrac{\partial}{\partial y} \\
	A & D
	\end{vmatrix} \\
	&=\dfrac{\partial}{\partial x} (D) - \dfrac{\partial}{\partial y} (A)\\
	&=\dfrac{\partial (x,y)}{\partial x} - \dfrac{\partial (x,y)}{\partial y} \\
	& = 1 - 1 \\
	&= 0
	\end{align*}
	\item Given $E = \langle y , x \rangle$ and $A = (0 , -1) \quad B = (1 , 1 )$
	\\ To find the work done on the path A to B , we need to parameterize the path with $r(t)$
	\begin{align*}
	&r(t) = ( 1- t) A + t B \\
	& r(t) = (1-t) (0 , -1) + t (1, 1) \\
	& r(t) = (0 , -1 + t) + ( t , t) \\
	& r(t) = (t, -1 + 2t)\quad
	\begin{cases}
	\text{for t } \in [0,1]
	\end{cases} \\
	&\dfrac{dr}{dt} = (1 , 2)
	& E = \langle y , x \rangle \\
	& E(t) = \langle -1 + 2t , t \rangle \\
	& E(t) \cdot \dfrac{dr}{dt} = \langle -1 + 2t , t \rangle \cdot \langle 1 , 2 \rangle\\
	& = -1 + 4t \\
	& W = \int_c - 1 + 4t~dt \quad \text{where c = [0,1]} \\
	& W = \int^1_0 -1 + 4t ~ dt \\
	& = \left[-t + 2t^ 2 \right]_0^1 \\
	& = \left[-1 + 2(1)^ 2 \right] - \left[-0 + 2(0)^ 2 \right] \\
	& = -1 + 2 = 1
	\end{align*}
\end{enumerate}


\textbf{Question (2a)}\\\
To show that the set of these vectors is a basis for $\mathbb{R}^3$\vspace{5mm}\\
$\begin{bmatrix}
1\\2\\3
\end{bmatrix}$, $\begin{bmatrix}
-2\\1\\0
\end{bmatrix}$, $\begin{bmatrix}
1\\0\\1
\end{bmatrix}$\vspace{5mm}\\
These two conditions must satisfy
\begin{enumerate}
	\item Linear Independence: Thus the vectors must be linearl independent
	\item Spanning: The vectors must span $\mathbb{R}^3$
\end{enumerate}
\textbf{Check for linear independence }\vspace{5mm}\\
$v_1=\begin{bmatrix}
1\\2\\3
\end{bmatrix}$, $v_2=\begin{bmatrix}
-2\\1\\0
\end{bmatrix}$ and $v_3=\begin{bmatrix}
1\\0\\1
\end{bmatrix}$\vspace{5mm}\\
are linearly independent if the only solution to the equation\\
$c_1v_1 + c_2v_2 + c_3v_3 = 0 $\\
is $c_1=c_2=c_3 = 0 $, where $0 =$ $\begin{bmatrix}
	0\\0\\0
\end{bmatrix}$\\
\vspace{5mm}\\
$c_1\begin{bmatrix}
1\\2\\3
\end{bmatrix}$+ $c_2\begin{bmatrix}
-2\\1\\0
\end{bmatrix}$+ $c_3\begin{bmatrix}
1\\0\\1
\end{bmatrix}$= $\begin{bmatrix}
0\\0\\0
\end{bmatrix}$\vspace{5mm}\\
$c_1 - 2c_2  + c_3 = 0$\\
$2c_1 + c_2  = 0$\\
$3c_1  + c_3 = 0$\vspace{5mm}\\
In a matrix form,  we have\vspace{5mm}\\
$\begin{pmatrix}
1&-2&1\\ 2&1&0\\ 3&0&1
\end{pmatrix}$$\begin{pmatrix}
c_1\\c_2\\c_3
\end{pmatrix}$ = $\begin{pmatrix}
0\\0\\0
\end{pmatrix}$\vspace{5mm}\\
The matrix coefficient is:\vspace{5mm}\\
A = $\begin{pmatrix}
1&-2&1\\ 2&1&0\\ 3&0&1
\end{pmatrix}$\vspace{5mm}\\
detA = 1$\begin{vmatrix}
1&0\\ 0&1
\end{vmatrix}$ - (-2)$\begin{vmatrix}
2&1\\ 3&0 \end{vmatrix}$ + 1$\begin{vmatrix}
2&1\\ 3&0
\end{vmatrix}$ = 1(1 - 0) + 2(2 - 0) + 1(0 - 3)\vspace{5mm}\\
 detA =1 + 4 - 3 = 2 $\neq$ 0 \vspace{5mm}\\
 Since the detA = 2 $\neq$ 0, implies that the only solution to the system is $c_1=c_2=c_3 = 0 $ which proves linear independence. Hence the three vectors are linearly independent \vspace{5mm}\\ 
 \textbf{Check for Spanning}:\vspace{5mm}\\
 As far as the set contains three linearly independent vectors in $\mathbb{R}^3$ , and the dimension of $\mathbb{R}^3$ is 3, these vectors also span $\mathbb{R}^3$. Hence any vector in $\mathbb{R}^3$ can be written as a linear combination of $v_1$, $v_2$ and $v_3$.\\
 In conclusion, the set of vectors $v_1$, $v_2$ and $v_3$ is  a basis in $\mathbb{R}^3$ \vspace{5mm}\\
\textbf{(2b)}. Given A is invertible  and AD = I , where I =  identity matrix\\
To show that D = A$^{-1}$, multiple both sides of the \\
AD = I by A$^{-1}$ , from the left since A is invertible.\\
A$^{-1}$(AD) = A$^{-1}$I\\
(A$^{-1}$A)D = A$^{-1}$I , where A$^{-1}$A = I\\
ID = A$^{-1}$I , but ID = D and A$^{-1}$I = A$^{-1}$\\
$\Longrightarrow$ D = A$^{-1}$\\
Hence as shown.\vspace{5mm}\\
\textbf{(2c)}. Given $v_1=\begin{bmatrix}
1\\0\\0
\end{bmatrix}$, $v_2=\begin{bmatrix}
1\\1\\0
\end{bmatrix}$ and $v_3=\begin{bmatrix}
1\\1\\1
\end{bmatrix}$\vspace{5mm}\\
To determine if these vectors are linearly independent, we verify if the solution to the following equation \\
$c_1v_1 + c_2v_2 + c_3v_3 = 0 $\\
is $c_1=c_2=c_3 = 0 $, where $0 =$ $\begin{bmatrix}
0\\0\\0
\end{bmatrix}$\vspace{5mm}\\
$c_1\begin{bmatrix}
1\\0\\0
\end{bmatrix}$+ $c_2\begin{bmatrix}
1\\1\\0
\end{bmatrix}$+ $c_3\begin{bmatrix}
1\\1\\1
\end{bmatrix}$= $\begin{bmatrix}
0\\0\\0
\end{bmatrix}$\vspace{5mm}\\
Expanding into system of equations, we have\\
$c_1 - 2c_2  + c_3 = 0$..........(1)\\
$c_2 + c_3  = 0$ .............(2)\\
$c_3 = 0$ .............(3)\\
From eqn(3), $c_3$ = 0. Substituting the value of $c_3$ into eqn(2) gives;\\
$c_2 + 0  = 0$ \\
$\Longrightarrow$ $c_2  = 0$\\
Substituting $c_2$ and $c_3$ values into eqn(1) gives,\\
$c_1 + 0 + 0 = 0$ \\
$\Longrightarrow$ $c_1  = 0$\\
Since the only solution to the of equations is $c_1=c_2=c_3 = 0 $, it implies that $v_1$, $v_2$ and $v_3$ linearly independent.\vspace{5mm}\\
\textbf{(2d)}. Given V = \{(x, $\frac{1}{2}$)$|$ x is real number\}, the set V is a subspace in $\mathbb{R}^2$ if it satisfies the conditions below\\
1. The zero vero vector in V\\
2. Closed under addition\\
3. Closed under scalar multiplication\\
\underline{check for condition 1 (zero vector)}\vspace{5mm}\\
The zero vector in $\mathbb{R}^2$ = $\begin{bmatrix}
0\\0
\end{bmatrix}$. \vspace{5mm}\\
 For V to be a subspace , (0, $\frac{1}{2}$) must be in V, since V = (x, )$|$ x is real number , however , the second component of the zero vector is  0 , not $\frac{1}{2}$.Therefore, the zero vector (0,0) is not in V, hence the first condition failed.\vspace{5mm}\\
\underline{check for condition 2 (Closed under addition)}\vspace{5mm}\\
Suppose given two vectors in V, say $v_1$ = ($x_1$, $\frac{1}{2}$) and $v_2$ = ($x_2$, $\frac{1}{2}$)\\
where $x_1$ and $x_2$ are real numbers .\\
$v_1 + v_2$ = ($x_1$, $\frac{1}{2}$) + ($x_2$, $\frac{1}{2}$) = ($x_1 + x_2$, 1) \\
but the second component of the sum is 1, not $\frac{1}{2}$ hence $v_1 + v_2$ $\notin$ V\\
This implies that V is not closed under addition, hence the second condition failed.\vspace{5mm}\\
\underline{check for condition 3 (Closed under scalar multiplication)}\vspace{5mm}\\
Consider a vector $v$ = ($x$, $\frac{1}{2}$) $\in$ V and a scalar c $\in$ $\mathbb{R}$ \\
The scalar cv = c($x$, $\frac{1}{2}$) = ($cx$, $\frac{c}{2}$). where $\frac{c}{2}$ is generally not equal to $\frac{1}{2}$ except c = 1.\\
Hence, scalar multiplication does not necessarily produce a vector of the form ($x$, $\frac{1}{2}$). Therefore, V is not closed under scalar multiplication.\\
In conclusion, V is not a subspace in $\in$ $\mathbb{R}^2$ since all the condition for a vector to be a subspace failed.\vspace{8mm}\\
\textbf{Question 3}\vspace{8mm}\\
 Given intersections (A, B, C, D) and some
unknowns (u, x, y, v).\\
(3a). The valid assumptions needed to solve the problem are:
1. Conservation of flow: the number of cars entering an intersection equals the number leaving it (no car park, disappear, or are created at intersection).\\
2. All numbers shown in the diagram represent vehicles per hour.\\
3. The variable u,v,x and y represent unknown traffic flows that we need to determine.\\
4. Traffic flows are constant during rush hour.\\
5. All turns are prohibited (since arrows show one-way streets with striaght flow only)\vspace{5mm}\\
(3b). The system of equations based on my assumptions are:\\
For intersection A: 450 + u = 610 + v , (inflow = outflow)\\
u - v = 160 ...........(1)\\
For intersection B: 520 + v = 480 + x  \\
-v + x = 40 ............(2)\\
For intersection C: 390 + x = 600 + y \\
x - y = 210 .............(3)\\
For intersection D: 640 + y = 310 + u\\
-y + u = 330 .............(4)\vspace{5mm}\\ 
Therefore the system of equations are:\\
u - v = 160 ...........(1)\\
-v + x = 40 ............(2)\\
x - y = 210 .............(3)\\
-y + u = 330 .............(4)\vspace{5mm}\\ 
(3c).\\
v = u - 160 ...........(1)\\
x = v + 40 ............(2)\\
y = x - 210 .............(3)\\
u = y + 330 .............(4)\vspace{5mm}\\
substituting back,\\
u = (u - 160 + 40 - 210)+ 330 \\
$\Longrightarrow$ u = u + 0 \\
$\Longrightarrow$ 0 = 0 \\
The system is not solvable because we cannot determine the values of the variables u,v,x and y which represent the unknown traffic flows.\vspace{5mm}\\
(3d). Given that the number of cars (y) between intersection C and D is 200 \\
from eqn(3) in (3c) above,
\begin{align*}
&y = x- 210 \quad y = 200 \\
&x = 410 \\ \\
& x = 40 + v \\
& v = 370 \\ \\
& u = 160 + v \\
&u = 530 \\
\end{align*}
\begin{itemize}
	\item \underline{ The amount of vehicles arriving At intersection A is: }
	\begin{align*}
	450 + u = 450 + 530 \\
	&= 980\\
	\end{align*}
	\item \underline{The amount of vehicles arriving At intersection B is: }
	\begin{align*}
	520 + V = 520 + 370 \\
	&= 890\\
	\end{align*}
	\item \underline{The amount of vehicles arriving At intersection C is: }
	\begin{align*}
	390 + x = 390 + 410 \\
	&= 800\\
	\end{align*}
\end{itemize}



\textbf{(4a)}. A vector field F is conservative if there exist a scalar potential function $\phi$ such that F = $\nabla$$\phi$. thus the curl of F must be $\nabla$ $\times$ F = 0\\
The vector fields given are :\\
i. $F_1$ =$ (2y + z)i + (2x + z)j + (x + y)k$\\
ii. $F_2$ =$ (y^2 + z^2)i + (x^2 + z)j + (x^2 + y^2)k$

Curl of vector field $F_1$ \\
 $F_1$ =$ (2y + z)i + (2x + z)j + (x + y)k$\vspace{5mm}\\
 The curl is given as $\nabla$ $\times$ $F_1$\vspace{5mm}\\
 $\nabla$ $\times$ $F_1$ = $\begin{vmatrix}
 i&j&k\\ \frac{\partial}{\partial x}&\frac{\partial}{\partial y}&\frac{\partial}{\partial z}\\ $ 2y + z$&$2x + z$&$x + y$
 \end{vmatrix}$\vspace{5mm}\\
= $(\frac{\partial}{\partial y}($x + y$) - \frac{\partial}{\partial z}($2x + z$))i$ - $(\frac{\partial}{\partial x}($x + y$) - \frac{\partial}{\partial z}($2y + z$))j$ + $(\frac{\partial}{\partial x}($2x + z$) - \frac{\partial}{\partial y}($2y + z$))k$\\
= (1 - 1)i - (1 - 1)j + (2 - 2)k = 0i + 0j + 0k = 0\\
Hence $\nabla$ $\times$ $F_1$ = 0 which implies that $F_1$ is conservative.\vspace{5mm}\\
Curl of vector field $F_2$ \\
$F_2$ =$ (y^2 + z^2)i + (x^2 + z)j + (x^2 + y^2)k$\vspace{5mm}\\
$\nabla$ $\times$ $F_2$ = $\begin{vmatrix}
i&j&k\\ \frac{\partial}{\partial x}&\frac{\partial}{\partial y}&\frac{\partial}{\partial z}\\  y^2 + z^2&x^2 + z&x^2 + y^2
\end{vmatrix}$\vspace{5mm}\\
= $(\frac{\partial}{\partial y}(x^2 + y^2) - \frac{\partial}{\partial z}(x^2 + z))i$ - $(\frac{\partial}{\partial x}(x^2 + y^2) - \frac{\partial}{\partial z}(y^2 + z^2))j$ + $(\frac{\partial}{\partial x}(x^2 + z) - \frac{\partial}{\partial y}(y^2 + z^2))k$\\
= (2y - 1)i - (2x - 2z)j + (2x - 2y)k $\neq$ 0\\
Since the curl is not equal to zero , thus $\nabla$ $\times$ $F_2$ $\neq$ 0 implies that  $F_2$ is not conservative.\\
(4b). The potential function for the conservative field $F_2$ . Given that \\
$F_1$ =$ (2y + z)i + (2x + z)j + (x + y)k$\vspace{5mm}\\
We need tgo find a scalar potential function $\phi$(x,y,z) such that $\nabla$$\phi$ = $F_1$ that is,\\
$\frac{\partial \phi}{\partial x}$ = 2y + z ........(1)\\
$\frac{\partial \phi}{\partial y}$ = 2x + z ........(2)\\
$\frac{\partial \phi}{\partial z}$ = x + z ...........(3)\\
Integrating eqn(1) with respect to x gives,\\
$\phi$(x,y,z) = (2y + z)x + g(y,z) ........(4) where g(y,z) is the constant\\
Next, using $\frac{\partial \phi}{\partial y}$ = 2x + z and differentiate eqn(4) with respect to y\\
$\frac{\partial \phi}{\partial y}$ = 2x + $\frac{\partial g}{\partial y}$\\ set this equation to 2x + z\\
2x + $\frac{\partial g}{\partial y}$ = 2x + z\\
 $\Longrightarrow$ $\frac{\partial g}{\partial y}$ = z\\
 Integrating this equation with respect to y gives,\\
 g(y,z) = zy + h(z)\\ where h(z) is a constant.\\
 $\phi$(x,y,z) = (2y + z)x + zy + h(z) ........(5)\\
 use $\frac{\partial \phi}{\partial z}$ = x + z and differentiate eqn(5)with respect to z\\
 $\frac{\partial \phi}{\partial z}$ = x + y + $\frac{\partial h}{\partial z}$\\
 x + y + $\frac{\partial h}{\partial z}$ =  x + z 
 $\Longrightarrow$ $\frac{\partial h}{\partial z}$ = 0 \\
 Integrating with respect to z\\
 h(z) = c  , c is a constant.\\
  Therefore, the final potential function for the conservative $F_1$ is\\
  $\phi$(x,y,z) = (2y + z)x + zy + c = 2xy + xz + zy + c\vspace{8mm}\\
 \textbf{Question (5)}\\
 (5a). Given matrix A = $\begin{bmatrix}
 1&-1&0\\ -1&2&-1\\ 0&-1&1
 \end{bmatrix}$\vspace{5mm}\\
 The matrix A is a square  matrix because it consist of the same number of rows and columns. In addition, it is symmetric since $A^T$ = A, it is a real matrix since the entries are real numbers and it singular since the determinant is zero.\\
 (5b). Using the Gaussian elimination to for the rank, we have\vspace{5mm}\\
 A = $\begin{bmatrix}
 1&-1&0\\ -1&2&-1\\ 0&-1&1
 \end{bmatrix}$\vspace{5mm}\\
 $R_2 + R_1 \rightarrow R_2$\vspace{5mm}\\
 $\begin{bmatrix}
 1&0&0\\ -1&1&-1\\ 0&-1&1
 \end{bmatrix}$\vspace{5mm}\\
 $R_3 + R_2 \rightarrow R_3$\vspace{5mm}\\
 $\begin{bmatrix}
 1&0&0\\ -1&1&-1\\ 0&0&0
 \end{bmatrix}$\vspace{5mm}\\
 Since we have a row of zeros, the rank is determine by the non zero rows. \\
 Hence the Rank(A) = 2\\
 (5c). Using the charactistic equation , det(A - $\lambda$I) = 0 , where $\lambda$= eigenvalue and I = identity matrix =  $\begin{bmatrix}
 1&0&0\\ 0&1&0\\ 0&0&1
 \end{bmatrix}$\vspace{5mm}\\
 $\lambda$I = $\begin{bmatrix}
 \lambda&0&0\\ 0&\lambda&0\\ 0&0&\lambda
 \end{bmatrix}$\vspace{5mm}\\
 A - $\lambda$I = $\lambda$I = $\begin{bmatrix}
 1 - \lambda&-1&0\\ -1&2 - \lambda&-1\\ 0&-1&1 - \lambda
 \end{bmatrix}$\vspace{5mm}\\
$|$ A - $\lambda$I$|$ = (1 - $\lambda$)((2 - $\lambda$)(1 - $\lambda$) - 1) - (-1)(-1(1 - $\lambda$) - 0) + 0 = 0\\
$-\lambda^3$ + 4$\lambda^2$ - 3$\lambda$ = 0\\
$\lambda^3$ - 4$\lambda^2$ + 3$\lambda$ = 0\\
 $\lambda$($\lambda^2$ - 4$\lambda$ + 3) = 0\\
 $\lambda_1 = 0$ and ($\lambda^2$ - 4$\lambda$ + 3) = 0\\
 ($\lambda$ - 1)($\lambda$ - 3) = 0\\
 $\Longrightarrow$ $\lambda_2 = 1$ and $\lambda_3 = 3$\\
 The eigenvalues are $\lambda_1 = 0$, $\lambda_2 = 1$ and $\lambda_3 = 3$ \vspace{5mm}\\
 (5d). For $\lambda_1 = 0$ , we have \\
 (A - $\lambda_1$I)$v_1$ = 0\vspace{5mm}\\
 $\Longrightarrow$ (A - $\lambda_1$I)$V_1$ = $\begin{bmatrix}
 1 - 0&-1&0\\ -1&2 - 0&-1\\ 0&-1&1 - 0
 \end{bmatrix}v_1$ = $\begin{bmatrix}
 0\\0\\0
 \end{bmatrix}$ , let $v_1 = \begin{bmatrix}
 x\\y\\z
 \end{bmatrix}$\vspace{5mm}\\
 $\begin{bmatrix}
 1&-1&0\\ -1&2&-1\\ 0&-1&1
 \end{bmatrix}\begin{bmatrix}
 x\\y\\z
 \end{bmatrix}$ = $\begin{bmatrix}
 0\\0\\0
 \end{bmatrix}$\vspace{5mm}\\
 x - y = 0 .........(1)\\
 -x + 2y - z = 0 .........(2)\\
 -y + z = 0 .............(3)\\
 From eqn(3) y = z and from eqn(1) x = y\\
 let z = t \\
 $\Longrightarrow$ y = t and x = t \vspace{5mm}\\
 hence $v_1$ = $\begin{bmatrix}
 	x\\y\\z
 \end{bmatrix}$ =  $\begin{bmatrix}
 t\\t\\t
 \end{bmatrix}$ = t$\begin{bmatrix}
 1\\1\\1
 \end{bmatrix}$ = $\begin{bmatrix}
 1\\1\\1
 \end{bmatrix}$ \vspace{5mm}\\
 
 
 For $\lambda_2 = 1$ , we have \\
 (A - $\lambda_2$I)$v_2$ = 0\vspace{5mm}\\
 $\Longrightarrow$ (A - $\lambda_2$I)$V_2$ = $\begin{bmatrix}
 1 - 1&-1&0\\ -1&2 - 1&-1\\ 0&-1&1 - 1
 \end{bmatrix}v_2$ = $\begin{bmatrix}
 0\\0\\0
 \end{bmatrix}$ , let $v_2 = \begin{bmatrix}
 x\\y\\z
 \end{bmatrix}$\vspace{5mm}\\
 $\begin{bmatrix}
 0&-1&0\\ -1&1&-1\\ 0&-1&0
 \end{bmatrix}\begin{bmatrix}
 x\\y\\z
 \end{bmatrix}$ = $\begin{bmatrix}
 0\\0\\0
 \end{bmatrix}$\vspace{5mm}\\
 - y = 0 .........(1)\\
 -x + y - z = 0 .........(2)\\
 -y = 0 .............(3)\\
 From eqn(1) and eqn(3) y = 0 . substituting y = 0 into eqn (2)\\
 $\Longrightarrow$ -x = z\vspace{5mm}\\
 hence $v_2$ = $\begin{bmatrix}
 x\\y\\z
 \end{bmatrix}$ =  $\begin{bmatrix}
 x\\0\\-x
 \end{bmatrix}$ = x$\begin{bmatrix}
 1\\0\\-1
 \end{bmatrix}$ = $\begin{bmatrix}
 1\\0\\-1
 \end{bmatrix}$ 
 
 For $\lambda_3 = 3$ , we have \\
 (A - $\lambda_3$I)$v_3$ = 0\vspace{5mm}\\
 $\Longrightarrow$ (A - $\lambda_3$I)$V_3$ = $\begin{bmatrix}
 1 - 3&-1&0\\ -1&2 - 3&-1\\ 0&-1&1 - 3
 \end{bmatrix}v_3$ = $\begin{bmatrix}
 0\\0\\0
 \end{bmatrix}$ , let $v_3 = \begin{bmatrix}
 x\\y\\z
 \end{bmatrix}$\vspace{5mm}\\
 $\begin{bmatrix}
 -2&-1&0\\ -1&-1&-1\\ 0&-1&-2
 \end{bmatrix}\begin{bmatrix}
 x\\y\\z
 \end{bmatrix}$ = $\begin{bmatrix}
 0\\0\\0
 \end{bmatrix}$\vspace{5mm}\\
 - 2x - y = 0 .........(1)\\
 -x - y - z = 0 .........(2)\\
 -y -2z = 0 .............(3)\\
 From eqn(3) y = -2z . substituting y = -2z into eqn (2)\\
 $\Longrightarrow$ x = z\vspace{5mm}\\
 hence $v_3$ = $\begin{bmatrix}
 x\\y\\z
 \end{bmatrix}$ =  $\begin{bmatrix}
 z\\-2z\\z
 \end{bmatrix}$ = z$\begin{bmatrix}
 1\\-2\\1
 \end{bmatrix}$ = $\begin{bmatrix}
 1\\-2\\1
 \end{bmatrix}$ 
 Therefore, the eigenvectors are :\vspace{5mm}\\
 $v_1$ = $\begin{bmatrix}
 1\\1\\1
 \end{bmatrix}$, $v_2$ = $\begin{bmatrix}
 1\\0\\-1
 \end{bmatrix}$, and $v_3$ = $\begin{bmatrix}
 1\\-2\\1
 \end{bmatrix}$\vspace{5mm}\\
 (5e). Check for orthogonal.Two vectors are orthongonal if their dot production is 0.\vspace{5mm}\\
  $v_1 \cdotp v_2$ = $\begin{bmatrix}
  1&1&1
  \end{bmatrix}$ $\cdotp$$\begin{bmatrix}
  1\\0\\-1
  \end{bmatrix}$ = 1 + 0 - 1 = 0 \vspace{5mm}\\
  
  $v_1 \cdotp v_3$ = $\begin{bmatrix}
  1&1&1
  \end{bmatrix}$ $\cdotp$$\begin{bmatrix}
  1\\-2\\1
  \end{bmatrix}$ = 1 - 2 + 1 = 0 \vspace{5mm}\\
  
  $v_2 \cdotp v_3$ = $\begin{bmatrix}
  1&0&-1
  \end{bmatrix}$ $\cdotp$$\begin{bmatrix}
  1\\-2\\1
  \end{bmatrix}$ = 1 + 0 - 1 = 0 \vspace{5mm}\\
  Therefore the three eigenvectors are orthogonal
  
\end{document}